\documentclass[a4paper, 11pt]{article}

\usepackage[left=2cm, top=3cm, text={17cm, 24cm}]{geometry}
\usepackage[czech]{babel}
\usepackage[utf8]{inputenc}
\usepackage[IL2]{fontenc}
\usepackage[hyphens]{url}

\begin{document}

\begin{titlepage}
\begin{center}
\textsc{\Huge Vysoké učení technické v~Brně}\\
\textsc{\huge Fakulta informačních technologií}\\
\vspace{\stretch{0.382}}
\LARGE Typografie a~publikování\,--\,4. projekt\\
\Huge Bibliografické Citácie\\
\vspace{\stretch{0.618}}
\Large \today \hfill Tomáš Žigo  \newpage
\end{center}
\end{titlepage}

\section{\LaTeX}

\LaTeX je software na tvorbu a~písanie dokumentov. Software je zdarma a~voľne dostupný, podobne ako aj jeho zdroje. To znamená, že môžme čítať a~aj meniť všetko od jadra {\LaTeX}u po posledné rozšírovacie balíčky. Viac o~úvode do {\LaTeX}u je uvedené v~\cite{KottwitzStefan}.

Jadro systému tvorí prekladač jazyka {\TeX} spoločne s~nadstavbou {\LaTeX}. Vstupom je textový súbor vytvorený ľubovoľným editorom, vačšinou sa jedná o~súbor, ktorý má v~názve \texttt{.tex}. Podrobnejšie informácie o~práci systému sú popísané v~\cite{RybickaJiri}. Ako nainštalovať samotný {\LaTeX} je obsiahnuté v~\cite{BojkoPavel}.

Základné stavebné jednotky syntaxe {\LaTeX}u tvoria nasledovné tri zložky, tak ako je uvedené v~\cite{VavricekJan}:
\begin{itemize}
\item Povolené znaky
\item Príkazy
\item Prostredie
\end{itemize}	

V~{\LaTeX}e je možné pracovať aj s~textom v ~slovenskom alebo českom jazyku. Pre prácu s týmito jazykmi sú však použité modifikované programy alebo doplnkové balíky. Bližšie informácie ohľadom týchto jazykov v {\LaTeX}e sú dostupné v~\cite{MartinekDavid}. Dokonca existuje združenie CSTUG (Československé združenie užívateľov {\TeX}u), kde odborníci udržiavajú svoje webové stránky a~snažia sa software trochu viac priblížiť spoločnosti. Viac informácií v~\cite{VeselyJindrich}.

Spracovanie textu však má aj akési pravidlá. Pravidlá typu viac medzier tesne vedľa seba sa považujú za jednu, alebo že medzezery na začiatku riadku sú ignorované je vhodné poznať. Viac k~téme v~\cite{OlsakPetr}. 

Oproti systému \texttt{MS Word} dokáže {\LaTeX} jednoduchšie vkladať vlastné štýly, aj keď samotné vyhľadávanie štýlov je zložitejšie. Navyše systém {\LaTeX} dokáže rozdeliť dokument na viacero častí, kde pri druhom menovanom systéme je potrebné použiť externé aplikácie. O~ďalších rozdieloch sa píše v~\cite{VavreckovaSarka}. 

{\TeX} nebol vyvynutý na tvorbu grafických prác, ale aj na napriek tomu je tu možnosť s~ním vytvárať aj grafické objekty. Viac informácií pre vytváranie grafiky v~\cite{SunolFrancesc}.
V~{\LaTeX}e je možné vytvárať dokonca aj obálky. Tejto problematike sa venuje \cite{MerciadriLuca}.  

\newpage
\bibliographystyle{czechiso}
\renewcommand{\refname}{Literatúra}
\bibliography{proj4}

\end{document}

\documentclass[10pt, hyperref={unicode}]{beamer}
\usetheme{Antibes}

\usepackage[czech]{babel}
\usepackage[utf8]{inputenc}
\usepackage{times}
\usepackage{graphics}
\usepackage{graphicx}
\usepackage{amsthm, amssymb, amsmath}
\usepackage{hyperref}

\title{Typografie a~publikování\,--\,5.~projekt}
\subtitle{Konečné automaty}
\author{Tomáš Žigo\texorpdfstring{\\ xzigot00@stud.fit.vutbr.cz}{}}
\institute{Vysoké učení technické v~Brně\\
	Fakulta informačních technologií}
\date{\today}

\begin{document}
\begin{frame}
\titlepage
\end{frame}

\begin{frame}
\frametitle{Prehlad}
\transblindsvertical
\tableofcontents
\end{frame}

\section{Úvod}
\subsection{Definícia}
\begin{frame}
\frametitle{Definícia}
\begin{itemize}
\item<1-> Výpočtový model.
\item<2-> Skladá sa z~konečného mnnožstva stavov a~prechodov.
\item<3-> Aplikácia napríklad pri vyhľadávaní v~texte.
\end{itemize}
\end{frame}

\subsection{Príklad}
\begin{frame}
\transblindsvertical
\begin{itemize}
\item<1-> Príklad jednoduchého konečného automatu so vstupom "abbc":
\begin{itemize}
\item<2-> Zo vstupu sa pohneme smerom šípky "a", to znamená, že prvý symbol zo vstupu ide do stavu $q_1$
\item<3-> Na vstupe nám ešte ostalo "bbc", šípka pre vstup "b", smeruje späť do toho istého uzla. Toto sa zopakuje 2x.
\item<4-> Už nám ostal iba znak "c". Šípka pre vstup "c" nás posunie do stavu $q_2$.
\item<5-> Na konci sme sa ocitli v~stave $q_2$, ktorý je aj koncovým stavom.
\end{itemize}
\end{itemize}
\begin{center}
\includegraphics[scale=0.5]{pr_automat.eps}
\end{center}
\end{frame}

\subsection{Funkcia}
\begin{frame}
\transblindsvertical
\frametitle{Funkcia konečného automatu}
\begin{enumerate}
\item<1-> Automat sa nachádza v~počiatočnom stave.
\item<2-> Automatu je predložený konečný vstupý reťazec postavený nad jeho vstupnou abecedou.
\item<3-> Automat prečíta a~odoberie symbol zo vstupného reťazca.
\item<4-> Automat prevedie prechod na zaklade symbolu a~aktuálneho vnútorného stavu.
\item<5-> Body 2-4 sa opakujú, pokiaľ nie je celý vstupný reťazec prečítaný.
\end{enumerate}
\end{frame}

\section{DKA a~NKA}
\begin{frame}
\transblindsvertical
\frametitle{Modely}
\begin{itemize}
\item<1-> Základné modely sú:
\begin{itemize}
\item<2-> Deterministický konečný automat (DKA)
\item<3-> Nedeterministický konečný automat (NKA)
\end{itemize}
\end{itemize}
\begin{center}
\includegraphics<4->[scale=0.4]{automat.eps}
\end{center}
\end{frame}

\subsection{DKA - Deterministický konečný automat}
\begin{frame}
\transblindsvertical
\frametitle{Definícia DKA}
\begin{itemize}
\item Pätica ${\Sigma,K,q_0,\delta,F}$
\item Kde:
\begin{itemize}
\item $\Sigma$ je vstupná abeceda.
\item $K$ je konečná množina stavov.
\item $q_0$ je počiatočný stav.
\item $\delta$ : ${Q x \Sigma \longrightarrow Q}$ je prechodová funkcia.
\item $F$ je množina akceptačných stavov.
\end{itemize}
\end{itemize}
\end{frame}

\begin{frame}
\transblindsvertical
\frametitle{Konfigurácia DKA}
\begin{itemize}
\item Dvojica ${(q,w)\epsilon K~x \Sigma^*}$
\item $q$ je aktuálny stav automatu
\item $w$ je dosiaľ neprečítaná časť vstupného slova
\end{itemize}
\end{frame}

\subsection{NKA - Nedeterministický konečný automat}
\begin{frame}
\transblindsvertical
\frametitle{Definícia NKA}
\begin{itemize}
\item Je rovnaká pätica ako DKA: ${\Sigma,K,q_0,\delta,F}$
\item Rozdiely medzi DKA:
\begin{itemize}
\item NKA povoľuje prechody na $\epsilon$
\item Nový stav nie je pre každý prechod určený jednoznačne.
\end{itemize}
\end{itemize}
\end{frame}

\section*{Zdroje}
\begin{frame}{Použité zdroje}
\transblindsvertical
	\begin{thebibliography}{1}
		\bibitem[Konečný automat]{automat} Konečný automat Wiki
		\newblock \url{https://bit.ly/2HD3u9It}
		\bibitem[Konečný automat2]{automat} Konečný automat
		\newblock \url{https://matematika.cz/konecny-automat}
		\bibitem[Konečný automat3]{automat} Konečný automat VOHO
		\newblock \url{http://voho.eu/wiki/konecny-automat/}
	\end{thebibliography}
\end{frame}
\end{document}

\documentclass[a4paper,11pt,twocolumn]{article}

\usepackage[left=1.5cm, top=2.5cm, text={18cm,25cm}]{geometry}
\usepackage{times}
\usepackage[czech]{babel}
\usepackage[IL2]{fontenc}
\usepackage[utf8]{inputenc}
\usepackage{amsthm, amssymb, amsmath}

\theoremstyle{definition}
\newtheorem{definition}{Definice}
\newtheorem{sentence}{Věta}

\begin{document}

\begin{titlepage}
\begin{center}
\textsc{\Huge Fakulta informačních technologií\\[0.3em]
Vysoké učení technické v~Brně}\\
\vspace{\stretch{0.382}}
\LARGE Typografie a~publikování\,--\,2. projekt\\[0.4em]
Sazba dokumentů a~matematických výrazů\\
\vspace{\stretch{0.618}}
\Large 2018 \hfill         Tomáš Žigo (xzigot00)\newpage
\end{center}
\end{titlepage}

\section*{Úvod}
V~této úloze si vyzkoušíme sazbu titulní strany, matematic\-kých
vzorců, prostředí a~dalších textových struktur obvyklých
pro technicky zaměřené texty (například rovnice (\ref{rovnice_1})
nebo Definice 	\ref{definition_turing} na straně \pageref{definition_turing}). Rovněž si vyzkoušíme používání
odkazů \verb|\ref| a \verb|\pageref|.

Na titulní straně je využito sázení nadpisu podle optického
středu s~využitím zlatého řezu. Tento postup byl
probírán na přednášce. Dále je použito odrádkování se
zadanou relativní velikostí 0.4em a 0.3em.

\section{Matematický text}
Nejprve se podíváme na sázení matematických symbolů
a~výrazů v~plynulém textu včetně sazby definic a~vět s~využitím
balíku \texttt{amsthm}. Rovněž použijeme poznámku pod
čarou s~použitím příkazu \verb|\footnote|. Někdy je vhodné
použít konstrukci \verb|${}$|, která říká, že matematický text
nemá být zalomen.
\begin{definition}
\label{definition_turing}
Turingův stroj \emph{(TS) je definován jako šestice
tvaru} ${M=(Q,\Sigma,\Gamma,\delta,q_0,q_F)}$\emph{, kde:}
\begin{itemize}
\item $Q$ \emph{je konecná množina} vnitrních (rídicích) stavů,
\item $\Sigma$ \emph{je konečná množina symbolů nazývaná} vstupní abeceda, ${\Delta\notin\Sigma,}$
\item $\Gamma$ \emph{je konečná množina symbolů,} ${\Sigma\;\subset\;\Gamma,\;\Delta\;\in\;\Gamma}$\emph{, nazývaná} pásková abeceda,
\item $\delta: (Q\backslash\{q_F\})\times \Gamma \rightarrow Q \times(\Gamma \cup\{L, R\})$ \emph{,kde $L, R\notin\Gamma,$ je parciální} přechodová funkce,
\item $q_0$ \emph{je} počáteční stav, ${q_0 \in Q}$ \emph{a}
\item $q_F$ \emph{je} koncový stav, ${q_F \in Q}$.
\end{itemize}
\end{definition}
Symbol $\Delta$ značí tzv. \emph{blank} (prázdný symbol), který se vyskytuje na místech pásky, která nebyla ještě použita (může ale být na pásku zapsán i~později).

\emph{Konfigurace pásky} se skládá z nekonečného řetězce, který reprezentuje obsah pásky a~pozice hlavy na tomto řetězci. Jedná se o~prvek množiny ${\{\gamma\Delta^\omega\:|\:\gamma \in \Gamma^\ast\} \times \mathbb{N}.}$\footnote{Pro libovolnou abecedu $\Sigma$ je ${\Sigma^\omega}$ množina všech \emph{nekonečných} řetězců nad $\Sigma$, tj. nekonečných posloupností symbolů ze $\Sigma$. Pro připomenutí: ${\Sigma^\ast}$ je množina všech \emph{konečných} řetězcú nad $\Sigma$.} 
\emph{Konfiguraci pásky} obvykle zapisujeme jako ${\Delta x y z\underline{z}x\Delta...}$ (podtržení značí pozici hlavy).\emph{Konfiguraci stroje} je pak dána stavem řízení a~konfigurací pásky. Formálně se jedná o~prvek množiny ${Q \times \{\gamma\Delta^\omega\:|\:\gamma \in \Gamma^\ast\} \times \mathbb{N}.}$
\subsection{Podsekce obsahující větu a odkaz}
\begin{definition}
\label{definition_retezec}
Řetězec $w$ nad abecedou $\Sigma$ je přijat TS \emph{ $M$ jestliže $M$ při aktivaci z~počáteční konfigurace pásky ${\underline{\Delta}w\Delta...}$ a~počátečního vztahu $q_0$ zastaví přechodem do koncového stavu $q_F$, tj.${(q_0,\Delta w\Delta^\omega,0)\underset{M}{\overset{\ast}{\vdash}}(q_F, \gamma, n)}$ pro nějaké ${\gamma \in \Gamma^\ast}$ a~${n \in \mathbb{N}.}$}

\emph{Množinu ${L(M)=\{w\:|\:w}$ je přijat TS $M{\}\subseteq\Sigma^\ast}$ nazýváme} jazyk přijímaný TS ${M.}$
\end{definition}

Nyní si vyskoušíme sazbu vět a~důkazů opět s~použitím balíku \texttt{amsthm.}
\begin{sentence}
\emph{Třída jazyků, které jsou přijímány TS, odpovídá} rekurzivně vyčíslitelným jazykům.
\end{sentence}
\begin{proof}
V~důkaze vyjdeme z~Definice \ref{definition_turing} a~\ref{definition_retezec}.
\end{proof}
\section{Rovnice a odkazy}
Složitější matematické formulace sázíme mimo plynulý text. Lze umístit několik výrazů na jeden řádek, ale pak je třeba tyto vhodně oddělit, například příkazem \verb|\quad|.

$$
		\sqrt[i]{y^3_i}
		\quad
		\text{kde } x_i \text{ je } i \text{-té sudé číslo}
		\quad
		y_i^{2*y_i}\neq y_i^{y_i^{y_i}}
$$

V rovnici (\ref{rovnice_1}) jsou využity tři typy zátvorek s~různou explicitně definovanou velikostí.

\begin{eqnarray}
\label{rovnice_1}
 x & = & \bigg\{\Big[\big(a + b\big) * c\Big]^d \oplus 1 \bigg\} \\
y & = & \lim_{x\to\infty} \frac{\sin^2x + \cos^2x}{\frac{1}{\log_{10}x}} \nonumber
\end{eqnarray}

V~této větě vidíme, jak vypadá implicitní vysázení limity $ \lim_{n\to\infty} f(n) $ v~normálním
	odstavci textu. Podobně je to i~s~dalšími symboly jako $ \sum^n_{i=1}2^i $ či
	$ \bigcup_{A \in \mathcal{B}}A $. V~případě vzorců $ \lim\limits_{x\to\infty} f(n) $ a~$\sum\limits^n_{i=1}2^i$ jsme si vynutili méně úspornou sazbu příkazem \verb|\limits|.
\begin{eqnarray}
		\int\limits^b_a f(x) \, \mathrm{d}x & = & - \int^a_b g(x) \, \mathrm{d}x \\
		\overline{\overline{A \vee B}} & \Leftrightarrow & \overline{\overline{A} \wedge \overline{B}}
	\end{eqnarray}
\section{Matice}
Pro sázení matic se velmi často používá prostředí \texttt{array} a~závorky (\verb|\left|, \verb|\right|).

$$
	\left(
	\begin{array}{ccc}
	a + b & \widehat{\xi + \omega} & \hat{\pi} \\
	\vec a & \overleftrightarrow{AC} & \beta \\
	\end{array}
	\right)
	= 1 \Longleftrightarrow \mathbb{Q = R} 
$$

$$
		\mathbf{A} =
		\left\|
		\begin{array}{cccc}
			a_{11} & a_{12} & \ldots & a_{1n} \\
			a_{21} & a_{22} & \ldots & a_{2n} \\
			\vdots & \vdots & \ddots & \vdots \\
			a_{m1} & a_{m2} & \ldots & a_{mn}
		\end{array}
		\right\|
		= \left|
		\begin{array}{cc}
			t & u \\
			v & w
		\end{array}
		\right|
		= tw - uv
$$

\quad Prostředí \texttt{array} lze úspěšně využít i~jinde.

$$
	\binom{n}{k} =
	\left\{
	\begin{array}{ll}
	\frac{n!}{k!(n - k)!} & \text{pro } 0 \leq k\;\leq n \\
	0 & \text{pro } k\;< 0 \text{ nebo } k\;> n
	\end{array}
	\right.
$$
\section{Závěrem}
V~prípadě, že budete potřebovat vyjádřit matematickou
konstrukci nebo symbol a~nebude se Vám dařit jej nalézt
v~samotném \LaTeX u, doporučuji prostudovat možnosti balíku
maker \AmS-\LaTeX.
\end{document}
